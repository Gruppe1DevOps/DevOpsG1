\newpage
\section{Aufgabenstellung 3}

\subsection{Ausgangssituation}
Sie sind Teil eines Qualitätssicherungsteams in einem fiktiven Unternehmen, das eine E-Commerce-Plattform betreibt. Die Plattform bietet Kameraprodukte an, nutzt KI-gestützte Produktsuche, integriert verschiedene Backend-Systeme (z. B. SAP, AWS, NetSuite, HubSpot) und führt den Nutzer durch einen komplexen, serviceorientierten Kaufprozess.

Ein zentraler Bestandteil dieses Systems ist der in der beigefügten Prozessgrafik dargestellte End-to-End-Ablauf, vom Website-Besuch bis zur finalen Mail-Bestätigung. Die Architektur umfasst APIs, externe Services, Cloud-Provisionierung und ERP-Integration.

\subsection{Ziel}
Erarbeiten Sie als Gruppe eine ganzheitliche Teststrategie für dieses System. Die Strategie soll praktikabel sein, typische Herausforderungen im DevOps-Umfeld adressieren und die Integration verschiedener Testebenen, Tools und Umgebungen berücksichtigen.

\subsection{Bearbeitungsschwerpunkte}
Bitte erarbeiten Sie eine Ausarbeitung (max. 6 Seiten + Anhang), in der Sie folgende Punkte behandeln:

\subsubsection{1. Testumgebungsstabilisierung}
\begin{itemize}
    \item Wie stellen Sie in der Entwicklungs- und Integrationsphase eine stabile Testumgebung sicher, insbesondere angesichts externer Systeme (SAP, AWS, GPT, HubSpot etc.)?
    \item Wie kann Service-Virtualisierung oder Mocking zum Einsatz kommen?
    \item Welche Datenanforderungen bestehen (z. B. synthetische Testdaten, Daten-Maskierung)?
\end{itemize}

\subsubsection{2. Testarten und Abdeckung}
\begin{itemize}
    \item Welche Testarten (z. B. Unit-, API-, Integrations-, E2E-, Load-, Performance-, Security-tests) sind erforderlich, um:
    \begin{itemize}
        \item die funktionalen Anforderungen abzudecken?
        \item nicht-funktionale Anforderungen wie Performance, Security, Verfügbarkeit und Datenintegrität zu prüfen?
    \end{itemize}
    \item Wo und wie werden diese Tests innerhalb der CI/CD-Pipeline ausgeführt?
\end{itemize}

\subsubsection{3. Testeffizienz und Wartbarkeit}
\begin{itemize}
    \item Wie strukturieren Sie Tests, um gezielt auf Systemveränderungen (z. B. SAP-Upgrade) reagieren zu können?
    \item Wie nutzen Sie z. B. Impact Analysis, modulare Architekturen oder risikobasiertes Testen, um Wiederverwendbarkeit und Selektivität zu ermöglichen?
\end{itemize}

\subsubsection{4. Reporting \& Testtransparenz}
\begin{itemize}
    \item Wo und wie sollen Testergebnisse dokumentiert und ausgewertet werden (z. B. Dashboards, Logs, automatisierte Reports)?
    \item Wer sind die Stakeholder für das Reporting (Dev, QA, Ops, Management)?
\end{itemize}

\subsubsection{5. Toolauswahl und Integration}
\begin{itemize}
    \item Welche Testtools (open source und/oder kommerziell) schlagen Sie für die Umsetzung vor für z. B.:
    \begin{itemize}
        \item Testautomatisierung
        \item Performance-Testing
        \item Service-Virtualisierung
        \item Testdatenmanagement
        \item Reporting \& Testmanagement
    \end{itemize}
\end{itemize}

\subsection{Teamarbeit \& Rollenverteilung}
Versucht in der Ausarbeitung die folgenden Rollen einzunehmen:
\begin{itemize}
    \item QA-Architekt: übergreifendes Testkonzept \& Architektur
    \item Testanalyst: Spezifikation von Testfällen und Daten
    \item Tool-Integrator: Toolauswahl \& CI/CD-Verknüpfung
\end{itemize}
Die Aufteilung ist eine Empfehlung, aber keine Pflicht.

\subsection{Abzugebende Materialien}
\begin{itemize}
    \item Schriftliches Konzept (PDF, max. 6 Seiten ohne Anhang)
    \item Architektur- oder Ablaufdiagramm(e) zur Teststrategie
    \item Tabelle mit empfohlener Tool-Landschaft
\end{itemize}

\subsection{Vorgesehene Bearbeitungsdauer}
Gesamt in etwa 5 Stunden pro Person.

\subsection{Lernziele}
\begin{itemize}
    \item Eine effektive Teststrategie im Kontext von DevOps
    \item Den Einsatz moderner Testtools konzeptionell bewerten
    \item Risiken in komplexen Integrationslandschaften erkennen und adressieren
    \item Testgetriebene CI/CD-Prozesse planen und visualisieren
\end{itemize}
