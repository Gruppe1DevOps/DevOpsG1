\newpage
\section{Aufgabenstellung 1}

%% Gruppengröße die bereits geformten Teams
%% Alle verwendeten Quellen sind anzugeben.
%% Abzugeben bis zum 04.05.2025 um 23:59 Uhr
%% Abgabe = ein PDF-Dokument der gesamten Gruppe in folgendem Schema
                      
%%   SS2025_DevOps_ILV_Gruppe_[Gruppenbuchstabe]_Aufgabe_1 beispielsweise
%%   240223_DevOps_ILV_Gruppe_A_Aufgabe_1 für die Gruppe A

%% Bei Fragen kontaktiert mich gerne per E-Mail: stefan.taucher@fh-burgenland.at
%% Aufgabenstellung siehe Datei "Aufgabe.pdf".

\subsection{Teilbereich Build and Code}

\begin{enumerate}
    \item Welche bereits vorgestellten bzw. zusätzlich recherchierte Vorgehensmodelle ermöglichen ein
    schnelles Iterieren und somit die Möglichkeit Feedback zeitnah durch die jeweiligen Stakeholder
    einzubringen? Zusätzlich soll auch darauf eingegangen werden, weshalb die gewählten
    Vorgehensmodelle dies im Vergleich zu anderen Modellen ermöglichen. \\

	
    
    \item Beschreiben Sie mindestens 3 Praktiken aus Extreme Programming im Detail und den Einfluss
    die diese Praktik auf die Softwareentwicklung und dem Ergebnis haben.\\

    %% possibile quellens:
    %% https://www.it-agile.de/agiles-wissen/agile-entwicklung/was-ist-extreme-programming/
    %% https://www.agile-heroes.com/de/magazine/extreme-programming/
    %% https://asana.com/de/resources/extreme-programming-xp

    \item Lesen Sie den Artikel https://blog.crisp.se/wp-content/uploads/2012/11/SpotifyScaling.pdf .
    Diese Informationen fassen Sie bitte in 1 bis maximal 2 Seiten zusammen. \\

	

    \item Beschreiben Sie Git im Detail sowie mindestens 2 Features im Detail. \\
  
    %% possibile quellens:
    %% https://docs.github.com/en/get-started/using-git/about-git
    %% https://www.geeksforgeeks.org/git-features/
    %% https://www.simplilearn.com/tutorials/git-tutorial/what-is-git#features_of_git


    \item Welche qualitätssteigenden Maßnahmen kennen Sie? Beschreiben Sie 2 beliebige im Detail. \\


    %% possibile quellens:
    %% code review https://about.gitlab.com/topics/version-control/what-is-code-review/
    %% automated testing https://www.testdevlab.com/blog/automated-testing
    %% weitere Bsp wären Continuous Integration/Continuous Deployment (CI/CD), Code Standards, 
    %% Logging, Code Refactoring 
	
\end{enumerate}

\subsection{Teilbereich DevOps}

Das Unternehmen ABC Ad Tech stellt eine SaaS-Lösung für Kunden bereit mit denen ihre
Werbekampagnen verwaltet werden können.
Aktuell gibt es ein Entwicklungsteam namens "Ad-Dev" mit 5 Personen die gemeinsam die
Lösung entwickeln. Der Source Code ist hierzu in git abgelegt. Der Output des Build-Prozesses
(=Artefakt) wird auf dem Firmen-PC von einem speziellen Mitarbeiter erstellt und dann
händisch auf eine Netzwerkdateifreigabe kopiert.
Für die Kunden wird die Lösung aktuell durch das Team namens "Ad-Ops" betrieben.
Die beiden Teams arbeiten unabhängig voneinander.
Das Entwicklungsteam "Ad-Dev" liefert alle 3 Monate eine neue "Produktivversion" und alle
zwei Woche eine neue "Testversion". Beide werden von Team "Ad-Ops" bei Verfügbarkeit
eingespielt, d.h. es wird das Artefakt von der Netzwerkdateifreigabe herunterkopiert und dann
"händisch" eingespielt. Die "Testversion" kommt immer auf eine Staging-Umgebung, welche
vom Team "Ad-QS" getestet wird und anschließend wird das Feedback mittels einer
Besprechung an die Entwicklung rückgemeldet. Auch die "Produktivversion" wird zuerst auf der
Staging-Umgebung getestet und sobald die Freigabe seitens "Ad-QS" gegeben wird erfolgt die
Einspielung auf das Produktivsystem durch das Team "Ad-Ops". Bei Problemen und sonstigen
Auffälligkeiten wird vom Team "Ad-Ops" aus Kontakt mit dem Entwicklungsteam "Ad-Dev"
aufgenommen.\\

\textbf{Aufgaben:}
\begin{enumerate}
    \item Beschreibe kurz mögliche Probleme in der aktuellen Organisation und Arbeitsaufteilung.\\
    
%% mögliche Problems:
%% Manuelles deployn -- Fehleranfällig -- Mensch 
%% devs und ops arbeiten unabhängig voneinander - keine Kommunikation
%% Probleme werden erst adressiert wenn sie zufällig gefunden werden - kein Monitoring
%% alle 3 Monate wird ausgeliefert - keine Continuous Delivery


    \item Beschreibe die notwendigen Schritte um in dieser Organisation DevOps (bis einschließlich
    Continuous Delivery) einzuführen. \\

    %% mögliche Schritte:
    %% Verantwortungen neu definieren
    %% CI/CD Pipeline - Automatisierung des Builds und Deployment
    %% regelmäßige Meetings
    %% Testautomatisierung - unit tests, performance, ui,...
    %% Monitoring/ Logging

    
    \item Welche Schritte sind in welcher Reihenfolge notwendig? \\
    
    %% mögliche Reihenfolge:
    %% Testautomatisierung - unit tests, performance, ui,...
    %% CI/CD Pipeline - Automatisierung des Builds und Deployment
    %% Monitoring/ Logging
    %% Verantwortungen neu definieren
    %% regelmäßige Meetings


    \item Welche Tools werden hierzu benötigt? \\
    
    %% mögliche Tools:
    %% Jenkins
    %% Git
    %% Docker
    %% Kubernetes
    %% Prometheus
    %% Grafana
    
    \item Beachte das solche Änderung Schritt für Schritt eingeführt werden müssen damit diese
    erfolgreich sein können. Ebenso müssen die wesentlichen Stakeholder überzeugt werden.
    Identifiziere die wesentlichen Stakeholder und liefere ihnen Argumente, die sie davon
    überzeugen das die Einführung von DevOps Praktiken vorteilhaft ist. \\

    %% Stakeholder:
    %% Geschäftsführung
    %% Kunden
    %% Entwicklungsteam

    %% Argumente:
    %% weniger manuelle Arbeit -- weniger Fehler -- weniger Kosten und schnellere Lieferzeiten
    %% Risiko wird reduziert -- siehe oben
    %% Besseres Zusammenarbeiten 
    %% Codequali steigt
    %% leichtere Wartung 


\end{enumerate}