\newpage
\section{Aufgabenstellung 1 Taucher}

%% Gruppengröße die bereits geformten Teams
%% Alle verwendeten Quellen sind anzugeben.
%% Abzugeben bis zum 04.05.2025 um 23:59 Uhr
%% Abgabe = ein PDF-Dokument der gesamten Gruppe in folgendem Schema
                      
%%   SS2025_DevOps_ILV_Gruppe_[Gruppenbuchstabe]_Aufgabe_1 beispielsweise
%%   240223_DevOps_ILV_Gruppe_A_Aufgabe_1 für die Gruppe A

%% Bei Fragen kontaktiert mich gerne per E-Mail: stefan.taucher@fh-burgenland.at
%% Aufgabenstellung siehe Datei "Aufgabe.pdf".

\subsection{Teilbereich Build and Code}

\subsubsection{Vorgehensmodelle}

Fragestellung: Welche bereits vorgestellten bzw. zusätzlich recherchierte Vorgehensmodelle ermöglichen ein
schnelles Iterieren und somit die Möglichkeit Feedback zeitnah durch die jeweiligen Stakeholder
einzubringen? Zusätzlich soll auch darauf eingegangen werden, weshalb die gewählten
Vorgehensmodelle dies im Vergleich zu anderen Modellen ermöglichen.
\\

\paragraph{Scrum}



\paragraph{Extreme Programming}


\paragraph{Test-Driven Development}


\paragraph{Kanban}



\subsubsection{Extreme Programming}

Fragestellung: Beschreiben Sie mindestens 3 Praktiken aus Extreme Programming im Detail und den Einfluss
die diese Praktik auf die Softwareentwicklung und dem Ergebnis haben.

\paragraph{Pair Programming}
Beim Pair Programming arbeiten zwei Entwickler gemeinsam an einem Computer. Eine Person (der "Driver") schreibt den Code, während die andere Person (der "Navigator") den Code überprüft, Probleme identifiziert und strategische Entscheidungen trifft. Die Rollen werden regelmäßig gewechselt.

\textbf{Einfluss auf die Softwareentwicklung:}
\begin{itemize}
    \item \textbf{Verbesserte Codequalität:} Durch kontinuierliches Review werden Fehler früher erkannt und behoben.
    \item \textbf{Wissenstransfer:} Entwickler lernen voneinander, was zu einer breiteren Verteilung von Wissen im Team führt.
    \item \textbf{Bessere Lösungsansätze:} Durch die Kombination verschiedener Perspektiven entstehen oft kreativere und effizientere Lösungen.
    \item \textbf{Reduzierte technische Schulden:} Die kontinuierliche Überprüfung verhindert Abkürzungen und schlechte Praktiken.
\end{itemize}

\paragraph{Continuous Integration (CI)}
Bei der Continuous Integration werden Codeänderungen mehrmals täglich in ein gemeinsames Repository integriert. Nach jeder Integration werden automatisierte Tests durchgeführt, um sicherzustellen, dass die Änderungen keine Fehler verursachen.

\textbf{Einfluss auf die Softwareentwicklung:}
\begin{itemize}
    \item \textbf{Frühe Fehlererkennung:} Probleme werden unmittelbar nach ihrer Entstehung identifiziert, was die Behebungskosten drastisch reduziert.
    \item \textbf{Reduzierte Integrationszeit:} Durch häufige kleine Integrationen werden große, problematische Merges vermieden.
    \item \textbf{Höhere Softwarestabilität:} Die Software bleibt kontinuierlich in einem funktionsfähigen Zustand.
    \item \textbf{Schnelleres Feedback:} Entwickler erhalten unmittelbare Rückmeldung zu ihren Änderungen.
\end{itemize}

\paragraph{Test-Driven Development (TDD)}
Bei TDD werden Tests geschrieben, bevor der eigentliche Code implementiert wird. Der Entwicklungsprozess folgt einem "Red-Green-Refactor"-Zyklus: Zuerst wird ein fehlschlagender Test geschrieben (Red), dann wird gerade genug Code implementiert, um den Test zu bestehen (Green), und schließlich wird der Code verbessert, ohne seine Funktionalität zu ändern (Refactor).

\textbf{Einfluss auf die Softwareentwicklung:}
\begin{itemize}
    \item \textbf{Klarere Anforderungen:} Das Schreiben von Tests zwingt Entwickler, die Anforderungen genau zu verstehen.
    \item \textbf{Besseres Design:} TDD fördert modularen, entkoppelten Code, der leichter zu testen ist.
    \item \textbf{Umfassende Testabdeckung:} Jede Funktionalität wird durch Tests abgedeckt, was zu robusterer Software führt.
    \item \textbf{Dokumentation durch Tests:} Tests dienen als lebende Dokumentation, die zeigt, wie der Code verwendet werden soll.
    \item \textbf{Sicherheit bei Refactoring:} Entwickler können Code mit Vertrauen umgestalten, da Tests Regressionen aufdecken.
\end{itemize}

    %% possibile quellens:
    %% https://www.it-agile.de/agiles-wissen/agile-entwicklung/was-ist-extreme-programming/
    %% https://www.agile-heroes.com/de/magazine/extreme-programming/
    %% https://asana.com/de/resources/extreme-programming-xp


\subsubsection{Zusammenfassung von Artikel Spotify Scaling}

Aufgabenstellung:Lesen Sie den Artikel https://blog.crisp.se/wp-content/uploads/2012/11/SpotifyScaling.pdf.
Diese Informationen fassen Sie bitte in 1 bis maximal 2 Seiten zusammen


\subsubsection{Git Features}

Fragestellung: Beschreiben Sie Git im Detail sowie mindestens 2 Features im Detail.

\paragraph{Was ist Git?}
Git ist ein verteiltes Versionskontrollsystem, das 2005 von Linus Torvalds entwickelt wurde.
Es ermöglicht die Verwaltung von Änderungen an Dateien und Projekten,
wobei jeder Entwickler eine vollständige Kopie des Projekts und seiner Historie auf seinem lokalen System hat.
Git ist besonders für seine Geschwindigkeit, Datenintegrität und Unterstützung für nicht-lineare, verteilte Workflows bekannt \cite{github-git}.

\paragraph{Feature 1: Branching und Merging}
Git bietet ein leistungsfähiges Branching-System, das es ermöglicht, parallele Entwicklungslinien zu erstellen und zu verwalten.
Branches sind leichtgewichtige Zeiger auf einen bestimmten Commit. Entwickler können schnell neue Branches erstellen,
zwischen ihnen wechseln und sie zusammenführen. Dies ermöglicht:

    \begin{itemize}
        \item Isolierte Entwicklung neuer Features
        \item Experimentieren ohne Risiko für den Hauptcode
        \item Parallele Arbeit an verschiedenen Aspekten des Projekts
        \item Einfaches Zusammenführen von Änderungen durch Merging
    \end{itemize}

    \paragraph{Feature 2: Distributed Version Control}
    Git ist ein vollständig verteiltes System, was bedeutet, dass jeder Entwickler eine vollständige Kopie des Repositories besitzt. Dies bietet mehrere Vorteile:

    \begin{itemize}
        \item Offline-Arbeit ist möglich
        \item Schnelle Operationen durch lokale Ausführung
        \item Redundanz und Backup durch multiple Kopien
        \item Flexible Workflow-Möglichkeiten
        \item Keine zentrale Schwachstelle
    \end{itemize}




    %% possibile quellens:
    %% https://docs.github.com/en/get-started/using-git/about-git
    %% https://www.geeksforgeeks.org/git-features/
    %% https://www.simplilearn.com/tutorials/git-tutorial/what-is-git#features_of_git


\subsubsection{Qualitätssteigernde Maßnahmen}

Fragestellung: Welche qualitätssteigenden Maßnahmen kennen Sie? Beschreiben Sie 2 beliebige im Detail.

   %% possibile quellens:
    %% code review https://about.gitlab.com/topics/version-control/what-is-code-review/
    %% automated testing https://www.testdevlab.com/blog/automated-testing
    %% weitere Bsp wären Continuous Integration/Continuous Deployment (CI/CD), Code Standards, 
    %% Logging, Code Refactoring 


\subsection{Teilbereich DevOps}

\subsubsection{Aufgabenstellung}

Das Unternehmen ABC Ad Tech stellt eine SaaS-Lösung für Kunden bereit mit denen ihre
Werbekampagnen verwaltet werden können.
Aktuell gibt es ein Entwicklungsteam namens "Ad-Dev" mit 5 Personen die gemeinsam die
Lösung entwickeln. Der Source Code ist hierzu in git abgelegt. Der Output des Build-Prozesses
(=Artefakt) wird auf dem Firmen-PC von einem speziellen Mitarbeiter erstellt und dann
händisch auf eine Netzwerkdateifreigabe kopiert.
Für die Kunden wird die Lösung aktuell durch das Team namens "Ad-Ops" betrieben.
Die beiden Teams arbeiten unabhängig voneinander.
Das Entwicklungsteam "Ad-Dev" liefert alle 3 Monate eine neue "Produktivversion" und alle
zwei Woche eine neue "Testversion". Beide werden von Team "Ad-Ops" bei Verfügbarkeit
eingespielt, d.h. es wird das Artefakt von der Netzwerkdateifreigabe herunterkopiert und dann
"händisch" eingespielt. Die "Testversion" kommt immer auf eine Staging-Umgebung, welche
vom Team "Ad-QS" getestet wird und anschließend wird das Feedback mittels einer
Besprechung an die Entwicklung rückgemeldet. Auch die "Produktivversion" wird zuerst auf der
Staging-Umgebung getestet und sobald die Freigabe seitens "Ad-QS" gegeben wird erfolgt die
Einspielung auf das Produktivsystem durch das Team "Ad-Ops". Bei Problemen und sonstigen
Auffälligkeiten wird vom Team "Ad-Ops" aus Kontakt mit dem Entwicklungsteam "Ad-Dev"
aufgenommen.\\


\subsubsection{Mögliche Probleme in der aktuellen Organisation und Arbeitsaufteilung}

Fragestellung: Beschreibe kurz mögliche Probleme in der aktuellen Organisation und Arbeitsaufteilung
    
%% mögliche Problems:
%% Manuelles deployn -- Fehleranfällig -- Mensch 
%% devs und ops arbeiten unabhängig voneinander - keine Kommunikation
%% Probleme werden erst adressiert wenn sie zufällig gefunden werden - kein Monitoring
%% alle 3 Monate wird ausgeliefert - keine Continuous Delivery

\subsubsection{Notwendige Schritte um in dieser Organisation DevOps einzuführen}
Aufgabenstellung: Beschreibe die notwendigen Schritte um in dieser Organisation DevOps (bis einschließlich
Continuous Delivery) einzuführen.

    %% mögliche Schritte:
    %% Verantwortungen neu definieren
    %% CI/CD Pipeline - Automatisierung des Builds und Deployment
    %% regelmäßige Meetings
    %% Testautomatisierung - unit tests, performance, ui,...
    %% Monitoring/ Logging


\subsubsection{Reihenfolge der Schritte}

Fragestellung: Welche Schritte sind in welcher Reihenfolge notwendig?
    %% mögliche Reihenfolge:
    %% Testautomatisierung - unit tests, performance, ui,...
    %% CI/CD Pipeline - Automatisierung des Builds und Deployment
    %% Monitoring/ Logging
    %% Verantwortungen neu definieren
    %% regelmäßige Meetings



\subsubsection{Benötigte Tools}

Fragestellung: Welche Tools werden hierzu benötigt?
    %% mögliche Tools:
    %% Jenkins
    %% Git
    %% Docker
    %% Kubernetes
    %% Prometheus
    %% Grafana

\subsubsection{Wesentliche Stakeholder und Argumente}

Fragestellung: Beachte das solche Änderung Schritt für Schritt eingeführt werden müssen damit diese
erfolgreich sein können. Ebenso müssen die wesentlichen Stakeholder überzeugt werden.
Identifiziere die wesentlichen Stakeholder und liefere ihnen Argumente, die sie davon
überzeugen das die Einführung von DevOps Praktiken vorteilhaft ist.

    %% Stakeholder:
    %% Geschäftsführung
    %% Kunden
    %% Entwicklungsteam

    %% Argumente:
    %% weniger manuelle Arbeit -- weniger Fehler -- weniger Kosten und schnellere Lieferzeiten
    %% Risiko wird reduziert -- siehe oben
    %% Besseres Zusammenarbeiten 
    %% Codequali steigt
    %% leichtere Wartung