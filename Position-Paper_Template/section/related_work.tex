%% Je nach dem in welcher Sprache ihr euer Paper schreiben wollt, benutzt bitte entweder den Deutschen-Titel oder den Englischen 
%% (einfach aus- bzw. einkommentieren mittels '%')

%Deutsch
\section{Alt vs. Neu}
\label{sec:alt_vs_neu}

Die Entwicklung von Smart-Office-Lösungen hat in den letzten Jahren stark an Dynamik gewonnen, wobei bestehende 
Ansätze wie in \cite{ref01}{Towards a Cloud-Based Smart Office Solution for Shared Workplace Individualization},
vorrangig auf Effizienzsteigerung und Komfort für allgemeine Nutzer:innen abzielen. Im Gegensatz 
dazu adressiert der neue Prototyp gezielt, die Bedürfnisse neurodivergenter Personen (z.B. Menschen mit ADHS 
oder Autismus) sowie Burnout-Rückkehrer:innen – Gruppen, deren Anforderungen in der bisherigen Lösung nicht 
berücktsichtigt werden.

Das etablierte Konzept aus \cite{ref01}{Towards a Cloud-Based Smart Office Solution for Shared Workplace 
Individualization} fokussiert sich auf die Automatisierung von Shared Open Workspaces durch ein Buchungssystem, 
ergonomische Anpassungen und die Überwachung von Umweltfaktoren, wie Temperatur oder CO2-Werten via Workplace 
Environment Index (WEI). Diese Metrik dient jedoch primär der ersten Stufe einer Individualisierung, wobei diese
nicht auf psychosoziale Bedürfnisse abzielt.
Der neue Ansatz hingegen integriert multisensorische Echtzeit-Anpassungen, um gezielt Stressfaktoren für 
neurodivergente Personen zu reduzieren. Beispielsweise minimiert eine dynamische Lichtsteuerung fluoreszierende
Belastungen, die bei ADHS-Betroffenen zu Konzentrationsverlust führen können. Für Autist:innen werden visuelle
Tagespläne und Unterbrechungsfilter eingeführt, während Burnout-Rückkehrer:innen von klaren Priorisierungsfunktionen
profitieren.

Während das ursprüngliche System eine Kubernetes-basierte Mikrodienstarchitektur mit komplexer Orchestrierung
beschreibt, setzt der neue Prototyp auf schlanke Amazon Web Services (AWS). Durch den Verzicht auf Kubernetes 
werden Ressourcen gebündelt und die Skalierbarkeit erhöht. Sensoren steuern beispielsweise Licht nicht nur basierend
auf statischen Daten, sondern reagieren dynamisch auf Tageszeit, Nutzerprofile und externe Bedingungen.
Weiters werden Tools wie Axe oder WAVE genutzt, um WCAG-Konformität sicherzustellen. Dies ermöglicht beispielsweise eine 
barrierearme Nutzung des Dashboards für Menschen mit sensorischen Einschränkungen.
Zudem verbindet die Lösung technische IoT-Funktionen (z.B. adaptive Beleuchtung) mit psychologischen 
Präventionsstrategien, um eine ganzheitliche Arbeitsumgebung zu schaffen. 
Diese Intersektionalität – die Integration von Technologie, Psychologie und Ethik – ist ein Novum in der 
Smart-Office-Landschaft.


\begin{table}[htbp]
\centering
\caption{Übersicht der wichtigsten Neuerungen}
\label{tab:delta}
\begin{tabularx}{\textwidth}{lX}
\toprule
\textbf{Aspekt} & \textbf{Beschreibung} \\
\midrule
\textbf{Zielgruppenspezifität} & Gezielte Anpassung für neurodivergente Gruppen und Burnout-Rückkehrer:innen. \\
\midrule
\textbf{Echtzeit-Steuerung} & Sensoren passen Licht dynamisch an Tageszeit und Nutzerprofile an. \\
\midrule
\textbf{Intersektionalität} & Kombination von IoT, Psychologie und Ethik in einer Plattform. \\
\midrule
\textbf{Ethischer Fokus} & Datensouveränität (Opt-in) und Barrierefreiheit als Kernprinzipien. \\
\bottomrule
\end{tabularx}
\end{table}

Das ursprüngliche Smart-Office-Konzept legt einen soliden Grundstein für effiziente Shared Workspaces. Der
neue Prototyp baut darauf auf und erweitert den Fokus um inklusive und präventive Elemente. 
Es soll eine Plattform entstehen, die nicht nur Räume optimiert, sondern gezielt das Wohlbefinden und die 
Produktivität vulnerabler Gruppen fördert. Dieser Paradigmenwechsel unterstreicht die Notwendigkeit, 
technologische Lösungen stärker an menschlicher Diversität auszurichten für eine nachhaltige Zukunft der Arbeitswelt.

% Englisch
%\section{Related Work}

%%Ziel dieses Kapitels ist es, dem Leser zu zeigen, welche anderen verwandten Arbeiten im selben Universum / Forschungsfeld bereits Ergebnisse 
%%geliefert haben. Hier geht es darum, dass man zeigt, dass man das eigene Forschungsfeld kennt und die wichtigsten Arbeiten darin kurz zusammengefasst beschreibt. 

%%Genauso ist es wichtig, dass man folgenden Unterschied bzw. das DELTA herausarbeitet:
%%\begin{itemize}
%%	\item[] Your Work vs. Related Work
%%	\item[] ODER
%%	\item[] Was haben die anderen Arbeiten eben NICHT gemacht und darum möchten wir es in diesem Position Paper motivieren
%%\end{itemize}

%% Dieses Kapitel ist sehr wichtig, weil es aus der Einleitung und der Problembeschreibung heraus nochmal zeigt, dass es einen NEED gibt,
%% um dieses Problem zu lösen (weil es bis jetzt kein anderer getan hat).

%%Folgende Key-Points gibt es bei diesem Kapitel zu beachten:
%%\begin{itemize}
%%	\item mind. 10 wissenschaftliche Quellen (andere Papers oder facheinschlägige Bücher) sollen identifiziert und zusammengefasst dargestellt / beschrieben werden.
%% Dabei sollen diese wissenschaftlichen Quellen zitiert werden
%%	\item die wissenschaftlichen Quellen sollen sich zu den nicht wissenschaftlichen Quellen die Waage halten im Verhältnis von 2/3 wissenschaftliche Quellen zu 1/3 andere Quellen 
%%	\item folgende Datenbanken gelten als "gültige Datenbanken" um wissenschaftliche Quellen zu suchen:
%%\subitem ACM Digital Library: \href{https://dl.acm.org/}{https://dl.acm.org/}	
%%	\subitem IEEE Explore: \href{https://ieeexplore.ieee.org/Xplore/home.jsp}{https://ieeexplore.ieee.org/Xplore/home.jsp} 
%%	\subitem Scopus: \href{https://www.scopus.com/search/form.uri?display=basic\#basic}{https://www.scopus.com/search/form.uri?display=basic\#basic}
%%	\subitem Science Direct: \href{https://www.sciencedirect.com/search}{https://www.sciencedirect.com/search} 
%%	\subitem Springer Verlag: \href{https://link.springer.com/}{https://link.springer.com/}
%%	\subitem Scitepress Digital Library: \href{https://www.scitepress.org/AdvancedSearch.aspx?SearchCriteria=Papers}{https://www.scitepress.org/AdvancedSearch.aspx?SearchCriteria=Papers} 
%%\end{itemize}

%%\textbf{VORGABE}: dieses Kapitel soll genau eine A4-Seite benötigen!!!!!!!!!!!!!





